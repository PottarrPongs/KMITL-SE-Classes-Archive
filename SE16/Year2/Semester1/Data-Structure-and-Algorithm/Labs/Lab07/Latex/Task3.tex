\documentclass[12pt]{report} % You can use 'article' or 'book' class as well

\usepackage{amsmath}
\usepackage{amssymb}
\usepackage{multicol}

\begin{document}


\noindent Theepakorn Phayonrat 67011352

\section*{Task 3: Complexit Analysis on Recursion (Do not use master method)}

\subsection*{3.1 Show time complexity of $T(n) = 2T(\frac{n}{2}) + n$}

\subsection*{Solution}

\begin{multicols}{2} \notag

\begin{align}
    T(n) & = 2(T(\frac{n}{2})) + n \\
    & = 2(2T(\frac{n}{2^{2}}) + \frac{n}{2}) + n \\
    & = 2(2T(\frac{n}{2^{2}})) + n + n \\
    & = 2^{2}T(\frac{n}{2^{2}}) + 2n \\
    & = 2^{3}T(\frac{n}{2^{3}}) + 3n \\
    & = 2^{4}T(\frac{n}{2^{4}}) + 4n \\
    & . \\
    & (Continue k times) \\
    & . \\
    T(n) & = 2^{k}T(\frac{n}{k}) + kn \\
\end{align}

\columnbreak

\begin{align}
    T(1) & = 1 \\
    T(n) & = T(1) \\
    \frac{n}{2^{k}} & = 1 \\
    n & = 2^{k} \\
    k & = log_{2}n \\
    \\
    T(n) & = 2^{k}T(\frac{n}{k}) + kn \\
    & = n(T(1)) + nlog_{2}n \\
    & = n + nlog_{2}n \\
\end{align}

\end{multicols}


\subsection*{Answer}
\noindent \therefore T(n) & = O(nlog(n)) \\

\newpage

\noindent Theepakorn Phayonrat 67011352
\subsection*{3.2 Show time complexity of $T(n) = T(n - 1) + 1$}

\subsection*{Solution}

\begin{multicols}{2} \notag

\begin{align}
    T(n) & = (T(n - 2) + 1) + 1 \\
    & = T(n - 2) + 2 \\
    & = T(n - 3) + 3 \\
    & = T(n - 4) + 4 \\
    & . \\
    & (Continue k times) \\
    & . \\
    T(n) & = T(n - k) + k \\
\end{align}

\columnbreak

\begin{align}
    T(0) & = 1 \\
    \text{Assume}\ n - k & = 0 \\
    \therefore n = k \\
    \therefore T(n) = 1 + n \\
\end{align}

\end{multicols}


\subsection*{Answer}
\noindent \therefore T(n) & = O(n) \\

\end{document}
