\documentclass[34pt]{report} % You can use 'article' or 'book' class as well

\usepackage{graphicx} % For including images
\usepackage{enumitem}
\usepackage{hyperref}
\usepackage{listings}
\usepackage{tabularx}
\usepackage{minted}

\begin{document}

% Title page
\begin{titlepage}
	\centering
	\vspace*{1cm} % Adjusts vertical space for the image
	% Insert your image (use the actual path and filename of your image)
	\includegraphics[width=0.3\textwidth]{../images/KMITL Logo.png} % Adjust width as needed

	\vspace{1cm} % Vertical space after the image
	{\LARGE \textbf{Homework 3}} \\[0.5cm] % Title
	\vspace{0.5cm}
	{\large \textbf{Computer Archiecture and Organization}} \\[0.5cm]
	{\large \textbf{Sofware Engineering Program,}} \\[0.5cm]
	{\large \textbf{School of Computer Engineering, KMITL}} \\[4cm]
	{\Large 67011352 Theepakorn Phayonrat} \\[0.5cm] % Authors (Use \\ the separate authors)
\end{titlepage}

\chapter*{Hurkle Game}

\subsection*{How to compile and Run}

\begin{lstlisting}[language=Bash ,basicstyle=\footnotesize\ttfamily]
as hurkle.s -o hurkle.o
ld hurkle.0 -o hurkle
./hurkle
\end{lstlisting}

\section*{Design and Implementation Details}

\subsection*{Structure}

\begin{enumerate}
    \item \texttt{init\_game}
        \begin{itemize}
            \item \textbf{Purpose}: Sets up the initial game state.
            \item \textbf{Description}: Calls \texttt{init\_random} to
                seed PRNG, then generates two random coordinates using
                \texttt{generate\_one_random\_number}.
        \end{itemize}
    \item \texttt{init\_random}
        \begin{itemize}
            \item \textbf{Purpose}: Initialize seed for random.
            \item \textbf{Description}: Opens \texttt{/dev/urandom},
                get seed and initializes \texttt{PRNG}.
        \end{itemize}
    \item \texttt{generate_one_random_number}
        \begin{itemize}
            \item \textbf{Purpose}: Generate a random number.
            \item \textbf{Description}: Generates a random number and
                stores into \texttt{r0}.
        \end{itemize}
\newpage
    \item \texttt{game\_loop}
        \item \textbf{Purpose}: Control the main flow of the game.
        \item \textbf{Description}:
            \begin{enumerate}
                \item Check whether player guesses 10 time.
                    \begin{itemize}
                        \item \textbf{Yes}: End game with player lose.
                        \item \textbf{No}: Increment the guess count
                            and continue to get player input.
                    \end{itemize}
                \item Check whether player guess is correct.
                    \begin{itemize}
                        \item \textbf{Yes}: End game with player won.
                        \item \textbf{No}: Give hint then continue
                            looping until game end.
                    \end{itemize}
            \end{enumerate}
        \end{itemize}
    \item \texttt{get\_player\_guess}
        \begin{itemize}
            \item \textbf{Purpose}: Take player input to continue the game.
            \item \textbf{Description}: Takes player input and convert to
                2 integers to check the hurkle location.
        \end{itemize}
    \item \texttt{give\_feedback}
        \begin{itemize}
            \item \textbf{Purpose}: Help player find hurkle
            \item \textbf{Description}: Compares the X Y coordinates
                to give hint to player
                \begin{itemize}
                    \item Compares Y coordinates: prints "Too high!" or "Too low!"
                    \item Compares X coordinates: prints "Too far left!" or "Too far right!"
                    \item Calculates Manhattan distance: $|GX - HX| + |GY - HY|$. \\
                        If Manhattan distance \le 2, prints \texttt{"You are very close!"}
                \end{itemize}
        \end{itemize}
    \item \texttt{print\_string}
        \begin{itemize}
            \item \textbf{Purpose}: Print string as ASCII output
            \item \textbf{Description}: Uses \texttt{SVC} with 4 in
                \texttt{r7}.
        \end{itemize}
    \item \texttt{itoa}
        \begin{itemize}
            \item \textbf{Purpose}: Convert integer to ASCII.
            \item \textbf{Description}: Builds string backward and copies
                to buffer by repeatedly dividing that integer by 10.
        \end{itemize}
\end{enumerate}

\subsection*{Key Designs Choices}

\begin{enumerate}
    \item \textbf{Randomness}: Use \texttt{/dev/urandom} to random
        number for this program.
    \item \textbf{Parsing integer to ASCII from user input}: Since the
        game need only 2 of 1 digit number from 0 to 9, we can split
        by skipping whitespace, take those 2 input and subtract by
        \texttt{'0'} to get integer value of the input.
\end{enumerate}

\section*{Challenges faced with Solutions}

\begin{center}
    \setlength{\extrarowheight}{5pt} % Adds 5pt to each row's height
    \begin{tabular}{| m{7.5cm} | m{7.5cm} |}
        \hline
        \textbf{Challenge} & \textbf{Solution} \\[4pt]
        \hline
        ARM System Calls are complicate to understand if you do not have that much knowledge in this field & Go to \href{https://chromium.googlesource.com/chromiumos/docs/+/master/constants/syscalls.md}{here} and read the specifications. \\
        \hline
        We need to convert integer back to ASCII every time we output to the terminal. & Create \texttt{itoa} to convert back to ASCII. \\
        \hline
    \end{tabular}
\end{center}

\section*{Game Demo}
\link{https://youtu.be/1oMMeVF0yOg}

\end{document}
