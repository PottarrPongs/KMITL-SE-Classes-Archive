\documentclass[34pt]{report} % You can use 'article' or 'book' class as well

\usepackage{graphicx} % For including images
\usepackage{enumitem}
\usepackage{listings}
\usepackage{tabularray}
\usepackage{minted}

\begin{document}

% Title page
\begin{titlepage}
	\centering
	\vspace*{1cm} % Adjusts vertical space for the image
	% Insert your image (use the actual path and filename of your image)
	\includegraphics[width=0.3\textwidth]{../images/KMITL Logo.png} % Adjust width as needed

	\vspace{1cm} % Vertical space after the image
	{\LARGE \textbf{Homework 2}} \\[0.5cm] % Title
	\vspace{0.5cm}
	{\large \textbf{Computer Archiecture and Organization}} \\[0.5cm]
	{\large \textbf{Sofware Engineering Program,}} \\[0.5cm]
	{\large \textbf{School of Computer Engineering, KMITL}} \\[4cm]
	{\Large 67011352 Theepakorn Phayonrat} \\[0.5cm] % Authors (Use \\ the separate authors)
\end{titlepage}

\section*{Design Your Own 8-bit Assembly to Hex Instruction Set}

\subsection*{Objective:}

Design a single instruction set and define a format to represent each
instruction to hexadecimal. Then, write sample programs using assembly,
and convert them to hex codes suitable for keying into a simulated SBC
(Single Board Computer).

\subsection*{System Specifications:}

\begin{itemize}
    \item CPU: 8-bit
    \item Registers: R0 top R10 (R0 is the accumulator)
    \item Instruction Size: 3 bytes (24 bits)
    \begin{itemize}
        \item Byte 1: Opcode
        \item Byte 2: Operand 1 (e.g., register)
        \item Byte 3: Operand 2 (register, address, or immediate)
    \end{itemize}
\end{itemize}

\subsection*{Required Instructions:}

\begin{tabular}{ l @{\hspace{4cm}} l }
    \vspace{0.15cm}
    Mnemonic & Description \\
    LD & Load immediateor from memory \\
    ST & Store accumulator to memory \\
    ADD & Add register or immediate to R0 \\
    SUB & Subtract register or immediate \\
    SHL & Shift R0 left by 1 bit \\
    SHR & Shift R0 right by 1 bit \\
    BR & Unconditional branch \\
    BRZ & Branch id zero (R0 == 0) \\
    BRG & Branch id zero (R0 $>$ 0) \\
    JSR & Jump to subroutine \\
    RET & Return from subroutine \\
\end{tabular}

\vspace{0.5cm}

\noindent You may add 1-2 extra instructions and explain their purpose.

\newpage

\subsection*{Part 1: Instruction Encoding}

\noindent 1. Define your own opcode mapping. Example:

\vspace{0.2cm}

\begin{tabular}{ l @{\hspace{4cm}} l }
    \vspace{0.15cm}
    Mnemonic & Description \\
    LD & 0x01 \\
    ST & 0x02 \\
\end{tabular}

\vspace{0.2cm}

\noindent 2. Define instruction format. For example:

\vspace{0.2cm}

\begin{tabular}{ l l l }
    \vspace{0.15cm}
    LD R0, \#12 & \rightarrow & 01 00 0C \\
    ADD R0, \#20 & \rightarrow & 03 00 14 \\

\end{tabular}

\subsection*{Part 2: Sample Assembly Program}

\noindent Write an assembly program ($\sim$ 10 instructions) that:

\begin{itemize}
    \item Load a number into R0
    \item Add another number
    \item Stores result in memory
    \item Checks result and branches if $>$ 0
    \item Calls a subroutine to clear R0
    \item Returns to main program
\end{itemize}
\vspace{0.2cm}

\subsection*{Part 3: Hex Code Conversion}

\noindent Convert your program into hex. Foer example:
\vspace{0.2cm}

\begin{tabular}{ l @{\hspace{4cm}} l }
    \vspace{0.15cm}
    LD R0, #12 & ; 01 00 0C\\
\end{tabular}

\subsection*{Deliverables:}
\begin{enumerate}
    \item Instruction set table with opcodes
    \item Assembly program(approx. 10 lines)
    \item Hexadecimal representation of program
    \item A step-by-step explanation of what each instruction does during the \textbf{fetch-decode-execute-store} cycle.
    \item Explanation of anny additional instructions \\

\end{enumerate}

\newpage

\subsection*{Part 1.0 Answer:}

Added 2 instructions \\

\begin{tabular}{ l @{\hspace{4cm}} l }
    \vspace{0.15cm}
    Mnemonic & Description \\
    LD & Load immediateor from memory \\
    ST & Store accumulator to memory \\
    ADD & Add register or immediate to R0 \\
    SUB & Subtract register or immediate \\
    SHL & Shift R0 left by 1 bit \\
    SHR & Shift R0 right by 1 bit \\
    BR & Unconditional branch \\
    BRZ & Branch id zero (R0 == 0) \\
    BRG & Branch id zero (R0 $>$ 0) \\
    JSR & Jump to subroutine \\
    RET & Return from subroutine \\
    CMP & Compare 2 registers given as arguments \\
    SYS & System Call (Software Interupt) \\

\end{tabular}


\subsection*{Part 1.1 Answer:}

Assigned opcode to every instruction \\

\begin{tabular}{ l @{\hspace{4cm}} l }
    \vspace{0.15cm}
    Mnemonic & Opcode \\
    LD & 0x01 \\
    ST & 0x02 \\
    ADD & 0x03 \\
    SUB & 0x04 \\
    SHL & 0x05 \\
    SHR & 0x06 \\
    BR & 0x07 \\
    BRZ & 0x08 \\
    BRG & 0x09 \\
    JSR & 0x0A \\
    RET & 0x0B \\
    CMP & 0x0C \\
    SYS & 0x0D \\

\end{tabular}

\newpage

\subsection*{Part 1.2 Answer:}

\begin{tabular}{ l l l }
    \vspace{0.15cm}
    LD R0, \#12 & \rightarrow & 01 00 0C \\
    ST R0, \#12 & \rightarrow & 02 00 0C \\
    ADD R0, \#20 & \rightarrow & 03 00 14 \\
    SUB R0, \#20 & \rightarrow & 04 00 14 \\
    SHL R0, \#20 & \rightarrow & 05 00 14 \\
    SHR R0, \#20 & \rightarrow & 06 00 14 \\
    BR R0, \#20 & \rightarrow & 07 00 14 \\
    BRZ R0, \#20 & \rightarrow & 08 00 14 \\
    BRG R0, \#20 & \rightarrow & 09 00 14 \\
    JSR loop & \rightarrow & 0A 1F (If $loop$ subroutine is 31) \\
    RET \#1 & \rightarrow & 0B 01 \\
    CMP R0, R1 & \rightarrow & 0C 00 01 \\
    SYS 1 & \rightarrow & 0D 01 \\

\end{tabular}

\subsection*{Part 2 Answer:}

\inputminted[fontsize=\small, breaklines, breakanywhere, breakindent=1em]{txt}{./ComOrgHW2P2.s}

\subsection*{Part 3 Answer:}

\inputminted[fontsize=\small, breaklines, breakanywhere, breakindent=1em]{txt}{./ComOrgHW2P3.hex}

\end{document}
