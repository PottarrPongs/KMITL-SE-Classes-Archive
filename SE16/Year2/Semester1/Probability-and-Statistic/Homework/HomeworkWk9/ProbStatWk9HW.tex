\documentclass[12pt]{report} % You can use 'article' or 'book' class as well

\usepackage{graphicx} % For including images
\usepackage{amsmath}
\usepackage{amssymb}
\usepackage{multicol}
\usepackage{pgfplots}
\usepackage{tikz}

\begin{document}

% Title page
\begin{titlepage}
	\centering
	\vspace*{1cm} % Adjusts vertical space for the image
	% Insert your image (use the actual path and filename of your image)
	\includegraphics[width=0.3\textwidth]{../images/KMITL Logo.png} % Adjust width as needed

	\vspace{1cm} % Vertical space after the image
	{\LARGE \textbf{Week 9 Homework}} \\[0.5cm] % Title
	\vspace{0.5cm}
	{\large \textbf{Probability Model and Data Analysis}} \\[0.5cm]
    {\large \textbf{Software Engineering Program,}} \\[0.5cm]
	{\large \textbf{Department of Computer Engineering,}} \\[0.5cm]
	{\large \textbf{School of Engineering, KMITL}} \\[1cm]
    {\Large 67011352 Theepakorn Phayonrat} \\[0.5cm] % Authors (Use \\ the separate authors)
\end{titlepage}

\section*{Homework of Functtions of RVs}

\subsection*{Question 1:}

\noindent Suppose that a random variable, $X$, has an expected value
of 8 and a variance of 1.8. What is the expected value and variance of
$-0.4 + 3X$?

\subsection*{Solution}

\noindent We know that $E[X] = 8$.

\noindent Let $Y = -0.4 + 3X$.

\begin{equation} \notag
\begin{split}
    \therefore Y & = -0.4 + 3X \\
    \therefore E[Y] & = -0.4 + 3E[X] \\
    & = -0.4 + 3(8) \\
    & = -0.4 + 24 \\
    \therefore E[Y] & = 23.6 \\
    \\
\end{split}
\end{equation}

\noindent Now, we know that $E[Y] = 23.6$.

\begin{equation} \notag
\begin{split}
    \therefore Var[Y] & = Var[-0.4 + 3E[X]]; \text{a = 3}\\
    \therefore Var[Y] & = a^{2} Var[X]\\
    & = 3^{2}(1.8) \\
    & = 9(1.8) \\
    \therefore Var[Y] & = 16.2 \\
    \\
\end{split}
\end{equation}

\subsection*{Answer}

$\therefore$ Expected value and the variance of $-0.4 + 3X$ are 23.6 and 16.2, respectively. \\

\newpage

\subsection*{Question 2:}

\noindent Monitor three customer purchasing smartphones at the Apple IT
store and observe whether each buys an iPhone 12 Pro Max for 40,000 THB
or Samsung Galaxy S20 for 30,000 THB. The random variable $N$ is the
number of customers purchasing an iPhone 12 Pro Max. Assume $N$ has PMF \\
\\
\[
P_{N}[n] =
\begin{cases}
    0.4, n = 0 \\
    0.2, n = 1, 2, 3 \\
    0, n = otherwise \\
\end{cases}
\]
\\
\noindent $M$ THB is the amount of money paid by three customers. \\
1. Express $M$ as a function of $N$ \\
2. Find PMF of $M$ \\
3. Find $E[M]$ \\
4. Find $Var[M]$ \\


\subsection*{Solution}

\vspace{0.5cm}

\noindent 1.

\begin{equation} \notag
\begin{split}
    M & = 40000N + 30000(3 - N) \\
    & = (40000 - 30000)N + 3(30000) \\
    \therefore M & = 10000N + 90000 \\
\end{split}
\end{equation}

\vspace{0.5cm}

\noindent 2.

\vspace{0.5cm}
\[
P_{M}[m] =
\begin{cases}
    0.4, m = 90000 \\
    0.2, m = 100000, 110000, 120000 \\
    0, m = otherwise \\
\end{cases}
\]

\vspace{0.5cm}

\newpage

\noindent 3.

\vspace{0.5cm}

\noindent We know that $E[M] = 10000E[N] + 90000$ and $E[M] = \displaystyle\sum_{m}m \cdot P_{M}(m)$

\begin{equation} \notag
\begin{split}
    E[M] & = 90000(0.4) + 100000(0.2) + 110000(0.2) + 120000(0.2) \\
    & = 36000 + 20000 + 22000 + 24000 \\
    \therefore E[M] & = 102000 \text{ THB} \\
    \\
\end{split}
\end{equation}

\vspace{0.5cm}

\noindent 4.

\vspace{0.5cm}

\noindent To find $Var[M]$, we need to find $E[M^{2}]$. \\
\noindent We know that $E[M^{2}] = \displaystyle\sum_{m}m^{2} \cdot P_{M}(m)$ \\

\begin{equation} \notag
\begin{split}
    E[M^{2}] & = 90000^{2}(0.4) + 100000^{2}(0.2) + 110000^{2}(0.2) + 120000^{2}(0.2) \\
    \therefore E[M^{2}] & = 10.54 \times 10^{9} = 10540000000 \\
\end{split}
\end{equation}

\noindent Now, we can find $Var[M]$

\begin{equation} \notag
\begin{split}
    Var[M] & = E[M^{2}] - (E[M])^{2} \\
    \therefore Var[M] & = 10540000000 - 10404000000 = 136000000 \text{ THB}^{3} \\
\end{split}
\end{equation}

\end{document}
