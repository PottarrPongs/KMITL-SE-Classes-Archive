\documentclass[12pt]{report} % You can use 'article' or 'book' class as well

\usepackage{graphicx} % For including images
\usepackage{amsmath}
\usepackage{amssymb}
\usepackage{multicol}
\usepackage{pgfplots}
\usepackage{tikz}
\usepackage{pgfplots}
\pgfplotsset{compat=1.18}

\begin{document}

% Title page
\begin{titlepage}
	\centering
	\vspace*{1cm} % Adjusts vertical space for the image
	% Insert your image (use the actual path and filename of your image)
	\includegraphics[width=0.3\textwidth]{../images/KMITL Logo.png} % Adjust width as needed

	\vspace{1cm} % Vertical space after the image
	{\LARGE \textbf{Week 10 Homework}} \\[0.5cm] % Title
	\vspace{0.5cm}
	{\large \textbf{Probability Model and Data Analysis}} \\[0.5cm]
    {\large \textbf{Software Engineering Program,}} \\[0.5cm]
	{\large \textbf{Department of Computer Engineering,}} \\[0.5cm]
	{\large \textbf{School of Engineering, KMITL}} \\[1cm]
    {\Large 67011352 Theepakorn Phayonrat} \\[0.5cm] % Authors (Use \\ the separate authors)
\end{titlepage}

\section*{Homework of Continuous RVs}

\subsection*{Question 1:}

\noindent The random variable $X$ has a probability density function \\

\[
    f_{X}(x) =
    \begin{cases}
        cx & 0 \le x \le 2 \\
        0 & otherwise
    \end{cases}
\]

\noindent Use the PDF to find \\

\noindent (a) the constant $c$ \\
\noindent (b) $P[0 \le X \le 1]$ \\
\noindent (c) $P[-\frac{1}{2} \le X \le \frac{1}{2}]$ \\
\noindent (d) the CDF $F_{X}(x)$

\subsection*{Solution}

\noindent (a) \\

\begin{align*}
    \int_{0}^{2} c x \, dx & = 1 \\
    c \int_{0}^{2} x \, dx & = 1 \\
    \left c \frac{x^{2}}{2} \right|_{0}^{2} & = 1 \\
    2c - 0c & = 1 \\
    2c & = 1 \\
    \therefore c & = \frac{1}{2} \\
\end{align*}

\newpage

\noindent (b) \\

\begin{align*}
    P[0 \le X \le 1] & = \int_{0}^{1} \frac{1}{2}x \, dx \\
    & = \left \frac{x^{2}}{4} \right|_{0}^{1} \\
    \therefore P[0 \le X \le 1] & = \frac{1}{4} - \frac{0}{4} = \frac{1}{4} \\
\end{align*}

\noindent (c) \\

\begin{align*}
    P[-\frac{1}{2} \le X \le \frac{1}{2}] & = \int_{0}^{\frac{1}{2}} \frac{1}{2}x \, dx \text{ ; since } P[X \le 0] = 0 \\
    & = \left \frac{x^{2}}{4} \right|_{0}^{\frac{1}{2}} \\
    \therefore P[-\frac{1}{2} \le X \le \frac{1}{2}] & = \frac{1}{16} - \frac{0}{4} = \frac{1}{16} \\
\end{align*}

\noindent (d) \\

From $F_{X}(x) & = \int_{0}^{x} f_{X}(u) \, du$ \\

\begin{align*}
    F_{X}(x) & = \int_{0}^{\frac{1}{2}} \frac{1}{2}x \, dx = \left \frac{x^{2}}{4} \right|_{0}^{x} \\
\end{align*}


\[
    F_{X}(x) =
    \begin{cases}
        0 & x < 0 \\
        \frac{x^{2}}{4} & 0 \le x \le 2 \\
        1 & x > 2 \\
    \end{cases}
\]

\newpage

\subsection*{Question 2:}

\noindent The cumulative distribution function of random variable $X$ is \\

\[
    F_{X}(x) =
    \begin{cases}
        0 & x < -1 \\
        \frac{x + 1}{2} & -1 \le x < 1 \\
        1 & x \ge 1 \\
    \end{cases}
\]

\noindent Find the PDF $f_{X}(x)$ of $X$

\subsection*{Solution}

\begin{align*}
    f_{X}(x) & = \frac{d F_{X}(x)}{d x} \\
    & = \frac{d [\frac{x + 1}{2}]}{d x} \\
    & = \frac{1}{2} \\
\end{align*}

\[
    \therefore F_{X}(x) =
    \begin{cases}
        \frac{1}{2} & -1 \le x < 1 \\
        0 & otherwise \\
    \end{cases}
\]

\newpage

\section*{Homework of Continuous RVs-Expected Value and $Var[X]$}

\subsection*{Question}

\noindent The probability density of the random variable $Y$ is \\

\[
    f_{Y}(y) =
    \begin{cases}
        \frac{3y^{2}}{2} & -1 \le y \le 1 \\
        0 & otherwise
    \end{cases}
\]

\noindent Sketch the PDF and find the following: \\

\begin{align*}
    &(1) \quad \text{the expected value } E[Y] \quad &&(2) \quad \text{the second moment } E[Y^{2}] \\
    &(3) \quad \text{the variance } Var[Y] \quad &&(4) \quad \text{the standard deviation } \sigma_{Y} \\
\end{align*}

\subsection*{Solution}

\[
\begin{tikzpicture}
    \begin{axis}[
        axis lines=middle,
        xlabel={$y$},
        ylabel={$f_Y(y)$},
        domain=-2:2,
        ymin=0, ymax=1.8,
        samples=100,
        width=10cm,
        height=6cm,
        xtick={-2,-1,0,1,2},
        ytick={0,0.5,1,1.5},
    ]
        % f_Y(y) between -1 and 1
        \addplot[thick, blue, domain=-1:1] {(3/2)*x^2};

        % outside [-1,1] it's zero
        \addplot[thick, blue, domain=-2:-1] {0};
        \addplot[thick, blue, domain=1:2] {0};

        % vertical lines at -1 and 1 (to make it look like your sketch)
        \addplot[dashed] coordinates {(-1,0) (-1,1.5)};
        \addplot[dashed] coordinates {(1,0) (1,1.5)};
    \end{axis}
\end{tikzpicture}
\]

\newpage

\noindent (1)

\begin{align*}
    E[Y] & = \int_{-1}^{1} y(\frac{3y^{2}}{2}) \, dy \\
    & = \left \frac{3y^{4}}{8} \right|_{0}^{\frac{-1}{1}} \\
    \therefore E[Y] & = \frac{3}{8} - \frac{3}{8} = 0 \\
\end{align*}

\noindent (2)

\begin{align*}
    E[Y^{2}] & = \int_{-1}^{1} y^{2}(\frac{3y^{2}}{2}) \, dy \\
    & = \left \frac{3y^{5}}{10} \right|_{0}^{\frac{-1}{1}} \\
    & = \frac{3}{10} - (-\frac{3}{10}) \\
    \therefore E[Y^{2}] & = \frac{6}{10} = \frac{3}{5} \\
\end{align*}

\noindent (3)

\begin{align*}
    Var[Y] & = E[YY^{2}] - (E[Y])^{2} \\
    \therefore Var[Y] & = \frac{3}{5} - 0 = \frac{3}{5} \\
\end{align*}

\newpage

\noindent (4)

\begin{align*}
    \sigma_{Y} & = \sqrt{Var[Y]} \\
    \therefore \sigma_{Y} & = \sqrt{\frac{3}{5}} \\
\end{align*}

\end{document}
