\documentclass[12pt]{report} % You can use 'article' or 'book' class as well

\usepackage[left=1cm,right=1cm,top=1.5cm,bottom=1.5cm]{geometry}
\usepackage{graphicx} % For including images
% \usepackage{amsmath}
% \usepackage{amssymb}
\usepackage{multicol}

\usepackage{PLAMacros}

\begin{document}

% Title page
\begin{titlepage}
	\centering
	\vspace*{1cm} % Adjusts vertical space for the image
	% Insert your image (use the actual path and filename of your image)
	\includegraphics[width=0.3\textwidth]{./images/KMITL Logo.png} % Adjust width as needed

	\vspace{1cm} % Vertical space after the image
    {\LARGE \textbf{Lecture Notes}} \\[0.5cm] % Title
	\vspace{0.5cm}
	{\large \textbf{Linear Algebra}} \\[0.5cm]
    {\large \textbf{Software Engineering Program,}} \\[0.5cm]
	{\large \textbf{Department of Computer Engineering,}} \\[0.5cm]
	{\large \textbf{School of Engineering, KMITL}} \\[1cm]
	{\Large 67011352 Theepakorn Phayonrat}[0.5cm] % Authors (Use \\ the separate authors)
\end{titlepage}

\tableofcontents
\chapter{Lecture 1: Vector}

\section{Vector \& Linear Combinations}

\subsection{Linear Combination}

$$cv + dw = c \begin{bmatrix} 1 \\ 1 \end{bmatrix} + d \begin{bmatrix} 2 \\ 3 \end{bmatrix} = \begin{bmatrix} c + 2d \\ c + 3d \end{bmatrix}$$

\subsection{Column Vector}
$$v = \begin{bmatrix} v_{1} \\ v_{2} \end{bmatrix}$$

\noindent Where $v_{1}$ is the first component of $v$. \\
\noindent Where $v_{2}$ is the second component of $v$.



\chapter{Lecture 2: Vector (cont.)}

\section{Matrices}

\noindent \textbf{Matrices} are combinations of \textbf{Vectors}


$$u = \begin{bmatrix} 1 \\ 2 \\ 3 \end{bmatrix}$$
$$v = \begin{bmatrix} 4 \\ 5 \\ 6 \end{bmatrix}$$
$$A = \begin{bmatrix} 1 & 4 \\ 2 & 5 \\ 3 & 6 \end{bmatrix}$$

$A$ is a 3 by 2 \textbf{matrix}: $m = 3$ rows and $n = 2$ columns

Suppose $A$ is a system which input $x$ and output $b$


\chapter{Lecture 3: Elimination}

$\PLAMatrix{b}{
1 & 2 & 3 \\
4 & 5 & 6
}$
\[
\plotA{25cm}{25cm}{20}{5}{
    % \addplot3[->, thick, purple] coordinates {(0, 0, 0) (1, 2, 3)};
    % \plotxyxV{red}{(-2, 1, 10)}
    % \plotV{red}{2}{-11}{4}
    \plotDPV{blue}{-7}{10}{5}{V_{1}}
    \plotDPV{green}{11}{-8}{3}{V_{2}}
    \plotDPV{cyan}{-1}{-5}{-9}{V_{3}}
    \plotDPV{red}{1}{5}{7}{V_{0}}
}

\end{document}
