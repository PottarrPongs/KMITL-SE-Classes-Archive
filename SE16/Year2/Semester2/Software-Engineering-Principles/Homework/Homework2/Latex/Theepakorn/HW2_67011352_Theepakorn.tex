\documentclass[12pt]{report}

\usepackage{graphicx}
\usepackage{amsmath}
\usepackage{amssymb}
\usepackage{minted}
\usepackage{multicol}
\usepackage{pgfplots}
\usepackage{tikz}

\begin{document}

\begin{titlepage}
	\centering
	\vspace*{1cm}
	\includegraphics[width=0.3\textwidth]{../../../images/KMITL Logo.png} \\
	\vspace*{1cm}
	{\LARGE \textbf{Homework 2}} \\[0.5cm]
	\vspace*{0.5cm}
	{\large \textbf{Software Engineering Principle}} \\[0.5cm]
    {\large \textbf{Software Engineering Program,}} \\[0.5cm]
	{\large \textbf{Department of Computer Engineering,}} \\[0.5cm]
	{\large \textbf{School of Engineering, KMITL}} \\[1cm]
    {\Large 67011352 Theepakorn Phayonrat} \\[0.5cm]
\end{titlepage}

\chapter*{Features}

\section*{Gantt Chart}
\noindent Used for planning project plan and tasks duration or deadline.

\section*{Class Diagram}
\noindent Used for designing classes in the projects.

\section*{Interaction Diagram}
\noindent Used for designing how classes interact each others in the projects.

\section*{Markdown Renderer for the task assignment page}
\noindent How it works:
\begin{itemize}
    \item As mentioned earlier, we can use markdown to express
        the task, $\therefore$ we need a markdown renderer.
\end{itemize}
\noindent Implementation Approach:
\begin{itemize}
    \item Use QEngineWebView Module in PyQt.
\end{itemize}

\newpage


\section*{VS-Code Extension (OPTIONAL)}
\texttt{TODO} extension in VS-Code with better description for
the task and with team member(s) assigned to that task. \\
How it works:
\begin{itemize}
    \item If you have comment with \texttt{TODO} in the front,
        you can add description of the task in a different
        entry and also in a markdown file.
    \item If you want to add a person in charge for that task
        (OPTIONAL), you can use \texttt{@TEMP}, where
        \texttt{TEMP} can be either role or team member names.
    \item After saved, you can access the \texttt{TODO}
        description as you hover and click to inspect task in
        the comment.
\end{itemize}
Implementation Approach:
\begin{itemize}
    \item Scan through the file looking for comment with
        \texttt{TODO} in the front then keep the entry into the DB.
    \item We can edit the \texttt{TODO} description inside a
        external markdown file.
\end{itemize}

\section*{Page included in this homework}
\begin{itemize}
    \item \textbf{Task Assignment Page}: Page to edit \texttt{TODO} for
        task assignments with markdown supported for better view. User
        can choose whether to edit in the manual mode or external text
        editor and save file because it can also fetch from real
        \texttt{.md} files in the real program.
\end{itemize}

\newpage

\section*{Code:}

\subsection*{TaskAssignmentPage.py}

\inputminted[fontsize=\small, breaklines, breakanywhere, breakindent=1em]{py}{../../TaskAssignmentPage.py}

\newpage

\section*{Output:}
\subsection*{Task Assignment Page}
\includegraphics[width=\textwidth]{../images/TaskAssignmentPageSyncDefault.png} \\
\\
\includegraphics[width=\textwidth]{../images/TaskAssignmentPageManualDefault.png} \\
\\
\includegraphics[width=\textwidth]{../images/TaskAssignmentPageSyncEdited.png} \\
\\
\includegraphics[width=\textwidth]{../images/TaskAssignmentPageManualEdited.png} \\

\section*{External Plugin}
\begin{itemize}
    \item \texttt{pip install markdown}
\end{itemize}


\end{document}
